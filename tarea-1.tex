\documentclass{article}
\usepackage[mathletters]{ucs}
\usepackage[utf8x]{inputenc}
\usepackage[spanish]{babel}
\usepackage{mathpazo}
\usepackage{amsmath}
\usepackage{amssymb}
\usepackage{enumitem}

\DeclareMathOperator{\LAND}{LAND}
\DeclareMathOperator{\LOR} {LOR}
\DeclareMathOperator{\PAND}{PAND}
\DeclareMathOperator{\POR} {POR}
\newcommand\comp{\overline}
\newcommand\respuesta[1]{\item[#1.]\input{t1/#1}}
\newcommand\enunciado{\textit}
\newcommand\luka{Łukasiewicz}

\parindent 0pt

\title{Inferencia---Tarea 1}
\author{Roberto Bonvallet \texttt{<roberto.bonvallet@usm.cl>}}
\date{10 de abril de 2012}

\begin{document}
\maketitle

En las demostraciones siguientes,
siempre se supondrá que
\(0 ≤ a, b, c, d ≤ 1\).

\begin{enumerate}[
  label=\arabic*.,
  font=\LARGE\bfseries,%
  labelindent=-.5in,%
  leftmargin=0pt,%
  labelsep=1em%
]

  \respuesta{1-i}
  \respuesta{1-ii}
  \respuesta{2-i}
  \respuesta{2-ii}
  \respuesta{2-iii}
  \respuesta{2-iv}

\end{enumerate}

\end{document}
