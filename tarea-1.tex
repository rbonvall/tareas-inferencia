\documentclass{article}
\usepackage[utf8]{inputenc}
\usepackage[spanish]{babel}
\usepackage{mathpazo}
\usepackage{amsmath}
\usepackage{amssymb}
\usepackage{fullpage}

\DeclareMathOperator{\LAND}{LAND}
\DeclareMathOperator{\LOR} {LOR}
\DeclareMathOperator{\PAND}{LAND}
\DeclareMathOperator{\POR} {POR}

\parindent 0pt

\title{Inferencia---Tarea 1}
\author{Roberto Bonvallet}
\date{10 de abril de 2012}

\begin{document}
\maketitle

En las demostraciones siguientes,
siempre se supondrá que
\(0\le a, b, c, d\le 1\).

Verificar que el operador \(\PAND(a, b) = ab\)
cumple las propiedades de T-norma.
%
\begin{description}
  \item[Conmutatividad:]
    \(
      \PAND(a, b) =
      ab =
      ba =
      \PAND(b, a).
    \)
  \item[Asociatividad:]
    \(
      \PAND\bigl(a, \PAND(b, c)\bigr) =
      \PAND(a, bc) =
      a(bc) =
      (ab)c =
      \PAND(ab, c) =
      \PAND\bigl(\PAND(a, b), c\bigr).
    \)
  \item[1 es la identidad:]
    \(
      \PAND(a, 1) = a\cdot 1 = a.
    \)
  \item[Monotonía:] sean \(a\le c\), \(b\le d\).

\end{description}

Verificar que el operador \(\LAND(a, b) = \max\{0, a + b - 1\}\)
cumple las propiedades de T-norma.
%
\begin{description}
  \item[Conmutatividad:]
    \(
      \LAND(a, b) =
      \max\{0, a + b - 1\} =
      \max\{0, b + a - 1\} =
      \LAND(b, a).
    \)
  \item[Asociatividad:]
    \(
      \LAND\bigl(a, \LAND(b, c)\bigr) =
      \LAND\bigl(a, \max\{0, b + c - 1\}\bigr) =
      \max\bigl\{0, a + \max\{0, b + c - 1\}\bigl\} =
    \)
  \item[1 es la identidad:]
    \(
      \PAND(a, 1) = a\cdot 1 = a.
    \)
  \item[Monotonía:] sean \(a\le c\), \(b\le d\).

\end{description}

Verificar que el operador \(\POR(a, b) = a + b - ab\)
cumple las propiedades de S-norma.
%
\begin{description}
  \item[Conmutatividad:]
    \(
      \POR(a, b) =
      a + b - ab =
      b + a - ba =
      \POR(b, a).
    \)
  \item[Asociatividad:]
    \(
      \POR\bigl(a, \POR(b, c)\bigr) =
      \POR(a, b + c - bc) =
      a + (b + c - bc) - a(b + c - bc) =
      a + b + c - bc - ab - ac - abc =
      (a + b - ab) + c - (a + b - ab)c =
      \POR(a + b - ab, c) =
      \POR\bigl(\POR(a, b), c\bigr).
    \)
  \item[0 es la identidad:]
    \(
      \POR(a, 0) = a + 0 - a\cdot 0 = a.
    \)
  \item[Monotonía:] sean \(a\le c\), \(b\le d\).
\end{description}

Verificar que el operador \(\LOR(a, b) = \min\{a + b, 1\}\)
cumple las propiedades de S-norma.
%
\begin{description}
  \item[Conmutatividad:]
    \(
      \LOR(a, b) =
      \min\{a + b, 1\} =
      \min\{b + a, 1\} =
      \LOR(b, a).
    \)
  \item[Asociatividad:]
    \(
      \LOR\bigl(a, \LOR(b, c)\bigr) =
    \)
  \item[0 es la identidad:]
    \(
      \LOR(a, 0) =
      \min\{a, 1\} =
      a.
    \)
  \item[Monotonía:] sean \(a\le c\), \(b\le d\).
\end{description}

\end{document}
