\documentclass{article}
\usepackage[mathletters]{ucs}
\usepackage[utf8x]{inputenc}
\usepackage[spanish]{babel}
\usepackage{mathpazo}
\usepackage{amsmath}
\usepackage{amssymb}
\usepackage{fullpage}
\usepackage{enumitem}

\DeclareMathOperator{\LAND}{LAND}
\DeclareMathOperator{\LOR} {LOR}
\DeclareMathOperator{\PAND}{PAND}
\DeclareMathOperator{\POR} {POR}
\newcommand\comp{\overline}

\parindent 0pt

\title{Inferencia---Tarea 1}
\author{Roberto Bonvallet \texttt{<roberto.bonvallet@usm.cl>}}
\date{10 de abril de 2012}

\begin{document}
\maketitle

En las demostraciones siguientes,
siempre se supondrá que
\(0 ≤ a, b, c, d ≤ 1\).

\begin{enumerate}[
  label=\arabic*.,
  font=\LARGE\bfseries,%
  labelindent=-.5in,%
  leftmargin=0pt,%
  labelsep=1em%
]

  \item[1-i.]
    \enunciado{%
  Verificar que el operador \(\PAND(a, b) = ab\)
  cumple las propiedades de T-norma.
}
%
\begin{description}
  \item[Conmutatividad:]
    \(
      \PAND(a, b) =
      ab =
      ba =
      \PAND(b, a).
    \)
  \item[Asociatividad:]
    \(
      \PAND\bigl(a, \PAND(b, c)\bigr) =
      \PAND(a, bc) =
      a(bc) =
      (ab)c =
      \PAND(ab, c) =
      \PAND\bigl(\PAND(a, b), c\bigr).
    \)
  \item[1 es la identidad:]
    \(
      \PAND(a, 1) = a\cdot 1 = a.
    \)
  \item[Monotonía:] sean \(a ≤ c\), \(b ≤ d\).

\end{description}

\enunciado{%
  Verificar que el operador \(\LAND(a, b) = \max\{0, a + b - 1\}\)
  cumple las propiedades de T-norma.
}
%
\begin{description}
  \item[Conmutatividad:]
    \(
      \LAND(a, b) =
      \max\{0, a + b - 1\} =
      \max\{0, b + a - 1\} =
      \LAND(b, a).
    \)
  \item[Asociatividad:] consideremos todos los casos posibles.
    \begin{align*}
      \LAND\bigl(a, \LAND(b, c)\bigr)
      &= \LAND(a, \max\{0, b + c - 1\}) \\
      &= \begin{cases}
           \LAND(a, b + c - 1)  & \text{si } b + c ≥ 1 \\
           \LAND(a, 0)          & \text{si } b + c < 1 \\
         \end{cases} \\
      &= \begin{cases}
           \max\{0, a + b + c - 2\}  & \text{si } b + c ≥ 1 \\
           \max\{0, a - 1\}          & \text{si } b + c < 1 \\
         \end{cases} \\
      &= \begin{cases}
           0              & \text{si } b + c ≥ 1 \text{ y } a + b + c < 2 \\
           a + b + c - 2  & \text{si } b + c ≥ 1 \text{ y } a + b + c ≥ 2 \\
           0              & \text{si } b + c < 1 \\
         \end{cases} \\
    \end{align*}
    La restricción que define la única región en que la norma es no nula
    se puede reescribir como:
    \begin{equation*}
       b + c ≥ 1 \quad\text{y}\quad b + c ≥ 2 - a.
    \end{equation*}
    Como siempre se cumple que \(2 - a ≥ 1\),
    entonces la primera parte de la restricción es redundante.
    Por lo tanto:
    \begin{equation*}
      \LAND\bigl(a, \LAND(b, c)\bigr)
       = \begin{cases}
           a + b + c - 2  & \text{si } a + b + c ≥ 2 \\
           0              & \text{en cualquier otro caso.} \\
         \end{cases}
    \end{equation*}
    La expresión obtenida es simétrica
    con respecto a \(a\), \(b\) y \(c\).
    Por lo tanto,
    si intercambiamos las variables de entrada
    obtendremos el mismo resultado:
    \begin{equation*}
      \LAND\bigl(a, \LAND(b, c)\bigr) =
      \LAND\bigl(c, \LAND(a, b)\bigr).
    \end{equation*}
  \item[1 es la identidad:]
    \(
      \LAND(a, 1) = \max\{0, a + 1 - 1\} = \max\{0, a\} = a.
    \)
  \item[Monotonía:] sean \(a ≤ c\), \(b ≤ d\).

\end{description}



  \item[1-ii.]
    \enunciado{%
  Verificar que el operador \(\POR(a, b) = a + b - ab\)
  cumple las propiedades de S-norma.
}
%
\begin{description}
  \item[Conmutatividad:]
    \(
      \POR(a, b) =
      a + b - ab =
      b + a - ba =
      \POR(b, a).
    \)
  \item[Asociatividad:]
    \(
      \POR\bigl(a, \POR(b, c)\bigr) =
      \POR(a, b + c - bc) =
      a + (b + c - bc) - a(b + c - bc) =
      a + b + c - bc - ab - ac - abc =
      (a + b - ab) + c - (a + b - ab)c =
      \POR(a + b - ab, c) =
      \POR\bigl(\POR(a, b), c\bigr).
    \)
  \item[0 es la identidad:]
    \(
      \POR(a, 0) = a + 0 - a\cdot 0 = a.
    \)
  \item[Monotonía:] sean \(a ≤ c\), \(b ≤ d\).
\end{description}

\enunciado{%
  Verificar que el operador \(\LOR(a, b) = \min\{a + b, 1\}\)
  cumple las propiedades de S-norma.
}
%
\begin{description}
  \item[Conmutatividad:]
    \(
      \LOR(a, b) =
      \min\{a + b, 1\} =
      \min\{b + a, 1\} =
      \LOR(b, a).
    \)
  \item[Asociatividad:] consideremos todos los casos posibles.
    \begin{align*}
      \LOR\bigl(a, \LOR(b, c)\bigr)
      &= \LOR(a, \min\{b + c, 1\}) \\
      &= \begin{cases}
           \LOR(a, 1)     & \text{si } b + c ≥ 1 \\
           \LOR(a, b + c) & \text{si } b + c < 1 \\
         \end{cases} \\
      &= \begin{cases}
           \min\{a + 1, 1\}     & \text{si } b + c ≥ 1 \\
           \min\{a + b + c, 1\} & \text{si } b + c < 1 \\
         \end{cases} \\
      &= \begin{cases}
           1         & \text{si } b + c ≥ 1 \\
           a + b + c & \text{si } b + c < 1 \text{ y } a + b + c < 1 \\
           1         & \text{si } b + c < 1 \text{ y } a + b + c ≥ 1 \\
         \end{cases}
    \end{align*}
    El mismo argumento de la demostración
    de la asociatividad de \(\LAND\) se puede aplicar aquí:
    \(b + c < 1\) es redundante en la segunda regla,
    y por simetría
    \(\LOR\bigl(a, \LOR(b, c)\bigr) = \LOR\bigl(c, \LOR(a, b)\bigr)\).
  \item[0 es la identidad:]
    \(
      \LOR(a, 0) =
      \min\{a, 1\} =
      a.
    \)
  \item[Monotonía:] sean \(a ≤ c\), \(b ≤ d\).
\end{description}



  \item[2-i.]
    \enunciado{%
  Verificar si \(\comp{A ∩ B} = \comp{A} ∪ \comp{B}\)
  con las normas \(\PAND\) y \(\POR\).
}

Para un \(x ∈ X\) cualquiera,
definamos \(a = μ_A(x)\)
y \(b = μ_B(x)\).
\begin{align*}
     μ_{\comp{A ∩ B}}(x)
  &= 1 - μ_{A ∩ B}(x) \\
  &= 1 - \PAND(μ_A(x), μ_B(x)) \\
  &= 1 - \PAND(a, b) \\
  &= 1 - (a + b - ab) \\
  &= (1 - a)(1 - b) \\
  &= \POR(1 - a, 1 - b) \\
  &= μ_{\comp{A} ∪ \comp{B}}(x)
\end{align*}
Por lo tanto,
la propiedad sí es satisfecha por las normas probabilísticas.

\saltito

\enunciado{%
  Verificar si \(\comp{A ∩ B} = \comp{A} ∪\comp{B}\)
  con las normas \(\LAND\) y \(\LOR\).
}

Para un \(x ∈ X\) cualquiera,
definamos \(a = μ_A(x)\)
y \(b = μ_B(x)\).
\begin{align*}
     μ_{\comp{A ∩ B}}(x)
  &= 1 - μ_{A ∩ B}(x) \\
  &= 1 - \LAND(μ_A(x), μ_B(x)) \\
  &= 1 - \LAND(a, b) \\
  &= 1 - \max\{0, a + b - 1\} \\
  &= \begin{cases}
       1 - (a + b - 1) & \text{si } a + b ≥ 1 \\
       1 - 0           & \text{en otro caso} \\
     \end{cases} \\
  &= \begin{cases}
       2 - a - b & \text{si } 2 - (a + b) < 1 \\
       1         & \text{en otro caso} \\
     \end{cases} \\
  &= \min\{1, 2 - a - b\} \\
  &= \min\{1, (1 - a) + (1 - b)\} \\
  &= \LOR(1 - a, 1 - b) \\
  &= μ_{\comp{A} ∪ \comp{B}}(x)
\end{align*}
Por lo tanto,
la propiedad sí es satisfecha por las normas de \luka.

\saltito


  \item[2-ii.]
    \enunciado{%
  Verificar si \(A ∩ (B ∪ C) = (A ∩ B) ∪ (A ∩ C)\)
  con las normas \(\PAND\) y \(\POR\).
}

Para un \(x ∈ X\) cualquiera,
definamos \(a = μ_A(x)\),
\(b = μ_B(x)\) y
\(c = μ_C(x)\).
\begin{align*}
     μ_{A ∩ (B ∪ C)}(x)
  &= \PAND\Bigl(
       μ_A(x),
       \POR\bigl(μ_B(x), μ_C(x)\bigr)
     \Bigr) \\
  &= \PAND\bigl(a, \POR(b, c)\bigr) \\
  &= a(b + c - bc).
\end{align*}
Por otra parte:
\begin{align*}
     μ_{(A ∩ B) ∪ (A ∩ C)}(x)
  &= \POR\Bigl(
       \PAND\bigl(μ_A(x), μ_B(x)\bigr),
       \PAND\bigl(μ_A(x), μ_C(x)\bigr),
     \Bigr) \\
  &= \POR\bigl(
       \PAND(a, b),
       \PAND(a, c)
     \bigr) \\
  &= \POR(ab, ac) \\
  &= ab + ac - a^2 bc \\
  &= a(b + c - abc).
\end{align*}
Si elegimos \(μ_A\), \(μ_B\), \(μ_C\) y \(x_0\) tales que:
\begin{align*}
  μ_A(x_0) &= 0.5, \\
  μ_B(x_0) &= 1, \\
  μ_C(x_0) &= 1,
\end{align*}
entonces
\begin{align*}
     μ_{A ∩ (B ∪ C)}(x)         &= 0.5(1 + 1 - 1) = 0.5 \\
  ≠  μ_{(A ∩ B) ∪ (A ∩ C)}(x_0) &= 0.5(1 + 1 - 0.5) = 0.75.
\end{align*}
Este contraejemplo demuestra que
las normas probabilísticas no satisfacen la propiedad
\(A ∩ (B ∪ C) = (A ∩ B) ∪ (A ∩ C)\).


\saltito

\enunciado{%
  Verificar si \(A ∩ (B ∪ C) = (A ∩ B) ∪ (A ∩ C)\)
  con las normas \(\LAND\) y \(\LOR\).
}

Para un \(x ∈ X\) cualquiera,
definamos \(a = μ_A(x)\),
\(b = μ_B(x)\) y
\(c = μ_C(x)\).
\begin{align*}
     μ_{A ∩ (B ∪ C)}(x)
  &= \LAND\Bigl(
       μ_A(x),
       \LOR\bigl(μ_B(x), μ_C(x)\bigr)
     \Bigr) \\
  &= \LAND\bigl(a, \LOR(b, c)\bigr) \\
  &= \LAND(a, \min{1, b + c}) \\
\end{align*}

\saltito



  \item[2-iii.]
    Verificar si \(A ∪ \comp{A} = X\)
con las normas \(\PAND\) y \(\POR\).

Verificar si \(A ∪ \comp{A} = X\)
con las normas \(\LAND\) y \(\LOR\).



  \item[2-iv.]
    Verificar si \(A ∩ \comp{A} = ∅\)
con las normas \(\PAND\) y \(\POR\).

Verificar si \(A ∩ \comp{A} = ∅\)
con las normas \(\LAND\) y \(\LOR\).



\end{enumerate}

\end{document}
