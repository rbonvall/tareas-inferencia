Verificar que el operador \(\POR(a, b) = a + b - ab\)
cumple las propiedades de S-norma.
%
\begin{description}
  \item[Conmutatividad:]
    \(
      \POR(a, b) =
      a + b - ab =
      b + a - ba =
      \POR(b, a).
    \)
  \item[Asociatividad:]
    \(
      \POR\bigl(a, \POR(b, c)\bigr) =
      \POR(a, b + c - bc) =
      a + (b + c - bc) - a(b + c - bc) =
      a + b + c - bc - ab - ac - abc =
      (a + b - ab) + c - (a + b - ab)c =
      \POR(a + b - ab, c) =
      \POR\bigl(\POR(a, b), c\bigr).
    \)
  \item[0 es la identidad:]
    \(
      \POR(a, 0) = a + 0 - a\cdot 0 = a.
    \)
  \item[Monotonía:] sean \(a ≤ c\), \(b ≤ d\).
\end{description}

Verificar que el operador \(\LOR(a, b) = \min\{a + b, 1\}\)
cumple las propiedades de S-norma.
%
\begin{description}
  \item[Conmutatividad:]
    \(
      \LOR(a, b) =
      \min\{a + b, 1\} =
      \min\{b + a, 1\} =
      \LOR(b, a).
    \)
  \item[Asociatividad:] consideremos todos los casos posibles.
    \begin{align*}
      \LOR\bigl(a, \LOR(b, c)\bigr)
      &= \LOR(a, \min\{b + c, 1\}) \\
      &= \begin{cases}
           \LOR(a, 1)     & \text{si } b + c ≥ 1 \\
           \LOR(a, b + c) & \text{si } b + c < 1 \\
         \end{cases} \\
      &= \begin{cases}
           \min\{a + 1, 1\}     & \text{si } b + c ≥ 1 \\
           \min\{a + b + c, 1\} & \text{si } b + c < 1 \\
         \end{cases} \\
      &= \begin{cases}
           1         & \text{si } b + c ≥ 1 \\
           a + b + c & \text{si } b + c < 1 \text{ y } a + b + c < 1 \\
           1         & \text{si } b + c < 1 \text{ y } a + b + c ≥ 1 \\
         \end{cases}
    \end{align*}
    El mismo argumento de la demostración
    de la asociatividad de \(\LAND\) se puede aplicar aquí:
    \(b + c < 1\) es redundante en la segunda regla,
    y por simetría
    \(\LOR\bigl(a, \LOR(b, c)\bigr) = \LOR\bigl(c, \LOR(a, b)\bigr)\).
  \item[0 es la identidad:]
    \(
      \LOR(a, 0) =
      \min\{a, 1\} =
      a.
    \)
  \item[Monotonía:] sean \(a ≤ c\), \(b ≤ d\).
\end{description}

