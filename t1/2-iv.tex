\enunciado{%
  Verificar si \(A ∩ \comp{A} = ∅\)
  con las normas \(\PAND\) y \(\POR\).
}

Para un \(x ∈ X\) cualquiera, definamos \(a = μ_A(x)\).
\begin{align*}
     μ_{A ∩ \comp{A}}(x)
  &= \PAND\bigl(μ_A(x), μ_{\comp{A}}(x)\bigr) \\
  &= \PAND(a, 1 - a) \\
  &= a(1 - a) \\
  &= a - a^2.
\end{align*}
Si elegimos \(μ_A\) y \(x_0\) tales que \(μ_A(x_0) = 0.5\),
entonces:
\begin{equation*}
  μ_{A ∩ \comp{A}}(x_0) = 0.5 - 0.25 = 0.25 ≠ 0 = μ_∅(x_0).
\end{equation*}
Este contraejemplo demuestra que
la propiedad no es satisfecha por las normas probabilísticas.


\enunciado{%
  Verificar si \(A ∩ \comp{A} = ∅\)
  con las normas \(\LAND\) y \(\LOR\).
}

Para un \(x ∈ X\) cualquiera, definamos \(a = μ_A(x)\).
\begin{align*}
     μ_{A ∩ \comp{A}}(x)
  &= \LAND\bigl(μ_A(x), μ_{\comp{A}}(x)\bigr) \\
  &= \LAND(a, 1 - a) \\
  &= \max\{0, a + (1 - a) - 1\} \\
  &= \max\{0, 0\} = 0 = μ_∅(x).
\end{align*}
Por lo tanto,
la propiedad sí es satisfecha por las normas de \luka.
