\enunciado{%
  Verificar que el operador \(\PAND(a, b) = ab\)
  cumple las propiedades de T-norma.
}
%
\begin{description}
  \item[Conmutatividad:]
    \(
      \PAND(a, b) =
      ab =
      ba =
      \PAND(b, a).
    \)
  \item[Asociatividad:]
    \(
      \PAND\bigl(a, \PAND(b, c)\bigr) =
      \PAND(a, bc) =
      a(bc) =
      (ab)c =
      \PAND(ab, c) =
      \PAND\bigl(\PAND(a, b), c\bigr).
    \)
  \item[1 es la identidad:]
    \(
      \PAND(a, 1) = a\cdot 1 = a.
    \)
  \item[Monotonía:] sean \(a ≤ c\), \(b ≤ d\).

\end{description}

\enunciado{%
  Verificar que el operador \(\LAND(a, b) = \max\{0, a + b - 1\}\)
  cumple las propiedades de T-norma.
}
%
\begin{description}
  \item[Conmutatividad:]
    \(
      \LAND(a, b) =
      \max\{0, a + b - 1\} =
      \max\{0, b + a - 1\} =
      \LAND(b, a).
    \)
  \item[Asociatividad:] consideremos todos los casos posibles.
    \begin{align*}
      \LAND\bigl(a, \LAND(b, c)\bigr)
      &= \LAND(a, \max\{0, b + c - 1\}) \\
      &= \begin{cases}
           \LAND(a, b + c - 1)  & \text{si } b + c ≥ 1 \\
           \LAND(a, 0)          & \text{si } b + c < 1 \\
         \end{cases} \\
      &= \begin{cases}
           \max\{0, a + b + c - 2\}  & \text{si } b + c ≥ 1 \\
           \max\{0, a - 1\}          & \text{si } b + c < 1 \\
         \end{cases} \\
      &= \begin{cases}
           0              & \text{si } b + c ≥ 1 \text{ y } a + b + c < 2 \\
           a + b + c - 2  & \text{si } b + c ≥ 1 \text{ y } a + b + c ≥ 2 \\
           0              & \text{si } b + c < 1 \\
         \end{cases} \\
    \end{align*}
    La restricción que define la única región en que la norma es no nula
    se puede reescribir como:
    \begin{equation*}
       b + c ≥ 1 \quad\text{y}\quad b + c ≥ 2 - a.
    \end{equation*}
    Como siempre se cumple que \(2 - a ≥ 1\),
    entonces la primera parte de la restricción es redundante.
    Por lo tanto:
    \begin{equation*}
      \LAND\bigl(a, \LAND(b, c)\bigr)
       = \begin{cases}
           a + b + c - 2  & \text{si } a + b + c ≥ 2 \\
           0              & \text{en cualquier otro caso.} \\
         \end{cases}
    \end{equation*}
    La expresión obtenida es simétrica
    con respecto a \(a\), \(b\) y \(c\).
    Por lo tanto,
    si intercambiamos las variables de entrada
    obtendremos el mismo resultado:
    \begin{equation*}
      \LAND\bigl(a, \LAND(b, c)\bigr) =
      \LAND\bigl(c, \LAND(a, b)\bigr).
    \end{equation*}
  \item[1 es la identidad:]
    \(
      \LAND(a, 1) = \max\{0, a + 1 - 1\} = \max\{0, a\} = a.
    \)
  \item[Monotonía:] sean \(a ≤ c\), \(b ≤ d\).

\end{description}

