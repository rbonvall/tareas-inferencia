\enunciado{%
  \(R\) relación difusa en \(X × Y\),
  \(S, T\) relaciones difusa en \(Y × Z\).
  Verificar que \(S ⊆ T ⇒ R\circ S ⊆ R\circ T\).
}

\(R\circ S\) y \(R\circ T\) son relaciones en \(X × Z\).
Sus funciones de membresía son:
\begin{align*}
  μ_{R\circ S}(x, z) &= \max_{y ∈ Y}\bigl\{
    \min\{
      \underbrace{μ_R(x, y)}_{r(y)},
      \underbrace{μ_S(y, z)}_{s(y)}
    \}
  \bigr\}
  \\
  μ_{R\circ T}(x, z) &= \max_{y ∈ Y}\bigl\{
    \min\{
      \underbrace{μ_R(x, y)}_{r(y)},
      \underbrace{μ_T(y, z)}_{t(y)}
    \}
  \bigr\}
\end{align*}
Para \(x\) y \(z\) fijos,
definamos \(r(y)\), \(s(y)\) y \(t(y)\)
como se indica en las expresiones anteriores.

Si \(S ⊆ T\), entonces \(μ_S ≤ μ_T\), y para todo \(y ∈ Y\):
\begin{align*}
     s(y)
  &≤ t(y) \\
     \min\{r(y), s(y)\}
  &≤ \min\{r(y), t(y)\} \\
     \max_{y ∈ Y}\bigl\{\min\{r(y), s(y)\}\bigr\}
  &≤ \max_{y ∈ Y}\bigl\{\min\{r(y), t(y)\}\bigr\}.
\end{align*}
Luego, \(μ_{R\circ S} ≤ μ_{R\circ T}\),
y por lo tanto \(R\circ S ⊆ R\circ T\).

