\enunciado{%
  Verificar si \(A ∩ (B ∪ C) = (A ∩ B) ∪ (A ∩ C)\)
  con las normas \(\PAND\) y \(\POR\).
}

Para un \(x ∈ X\) cualquiera,
definamos \(a = μ_A(x)\),
\(b = μ_B(x)\) y
\(c = μ_C(x)\).
\begin{align*}
     μ_{A ∩ (B ∪ C)}(x)
  &= \PAND\Bigl(
       μ_A(x),
       \POR\bigl(μ_B(x), μ_C(x)\bigr)
     \Bigr) \\
  &= \PAND\bigl(a, \POR(b, c)\bigr) \\
  &= a(b + c - bc).
\end{align*}
Por otra parte:
\begin{align*}
     μ_{(A ∩ B) ∪ (A ∩ C)}(x)
  &= \POR\Bigl(
       \PAND\bigl(μ_A(x), μ_B(x)\bigr),
       \PAND\bigl(μ_A(x), μ_C(x)\bigr),
     \Bigr) \\
  &= \POR\bigl(
       \PAND(a, b),
       \PAND(a, c)
     \bigr) \\
  &= \POR(ab, ac) \\
  &= ab + ac - a^2 bc \\
  &= a(b + c - abc).
\end{align*}
Si elegimos \(μ_A\), \(μ_B\), \(μ_C\) y \(x_0\) tales que:
\begin{align*}
  μ_A(x_0) &= 0.5, \\
  μ_B(x_0) &= 1, \\
  μ_C(x_0) &= 1,
\end{align*}
entonces
\begin{align*}
     μ_{A ∩ (B ∪ C)}(x)         &= 0.5(1 + 1 - 1) = 0.5 \\
  ≠  μ_{(A ∩ B) ∪ (A ∩ C)}(x_0) &= 0.5(1 + 1 - 0.5) = 0.75.
\end{align*}
Este contraejemplo demuestra que
las normas probabilísticas no satisfacen la propiedad
\(A ∩ (B ∪ C) = (A ∩ B) ∪ (A ∩ C)\).



\enunciado{%
  Verificar si \(A ∩ (B ∪ C) = (A ∩ B) ∪ (A ∩ C)\)
  con las normas \(\LAND\) y \(\LOR\).
}

Para un \(x ∈ X\) cualquiera,
definamos \(a = μ_A(x)\),
\(b = μ_B(x)\) y
\(c = μ_C(x)\).
\begin{align*}
     μ_{A ∩ (B ∪ C)}(x)
  &= \LAND\Bigl(
       μ_A(x),
       \LOR\bigl(μ_B(x), μ_C(x)\bigr)
     \Bigr) \\
  &= \LAND\bigl(a, \LOR(b, c)\bigr) \\
  &= \LAND(a, \min{1, b + c}) \\
\end{align*}

