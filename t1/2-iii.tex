Verificar si \(A ∪ \comp{A} = X\)
con las normas \(\PAND\) y \(\POR\).

Para un \(x ∈ X\) cualquiera, definamos \(a = μ_A(x)\).
\begin{align*}
     μ_{A ∪ \comp{A}}(x)
  &= \POR\bigl(μ_A(x), μ_{\comp{A}}(x)\bigr) \\
  &= \POR(a, 1 - a) \\
  &= a + (1 - a) - a(1 - a) \\
  &= 1  - a + a^2.
\end{align*}
Si elegimos \(μ_A\) y \(x_0\) tales que \(μ_A(x_0) = 0.5\),
entonces:
\begin{equation*}
  μ_{A ∪ \comp{A}}(x_0) = 1 - 0.5 + 0.25 = 0.75 ≠ 1 = μ_X(x_0).
\end{equation*}
Este contraejemplo demuestra que
la propiedad no es satisfecha por las normas probabilísticas.


Verificar si \(A ∪ \comp{A} = X\)
con las normas \(\LAND\) y \(\LOR\).

Para un \(x ∈ X\) cualquiera, definamos \(a = μ_A(x)\).
\begin{align*}
     μ_{A ∪ \comp{A}}(x)
  &= \LOR\bigl(μ_A(x), μ_{\comp{A}}(x)\bigr) \\
  &= \LOR(a, 1 - a) \\
  &= \max\{a + (1 - a), 1\} \\
  &= \max\{1, 1\} = 1 = μ_X(x).
\end{align*}
Por lo tanto,
la propiedad sí es satisfecha por las normas de Łukasiewics.
